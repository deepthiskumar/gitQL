\subsection{Git} \label{ssec:git}
The query language is for repositories version controlled using git. Git is a popular {\info{cite}} distributed version control system (VCS). It provides commands that can be used to retrieve information about the history of the repository. {\tt git log} along with the combination of a multitude of options can be used to view the history and obtain some information out of it. \gql is evaluated against the existing git commands and scripts that can be used to achieve the same query result. Also, 
\subsection{Choice Calculus}\label{ssec:vp}

Choice calculus is a formal language that provides a general representation for variations.
A variational string consists of all the variations encoded in this structural representation.
%has a set of program variants and a set of decisions.
Consider a program that has a function {\tt twice} which returns twice the value passed. Following are the two implementations that the developer wants to use.
\begin{lstlisting}
int twice(int x){return x+x;}  //variant 1

int twice(int x){return 2*x;}  //variant 2
\end{lstlisting}
Both the variants can be combined into a variational string using the choice calculus representation.
\begin{lstlisting}
int twice(int x){return $A\langle$x+x$,$2*x$\rangle$;}
\end{lstlisting}
%VPs can be represented as choice calculus expressions with binary choices and global scoped dimensions.\cite{erwig2011choice,erwig2015}
The variants of this variational string are obtained by selecting one of the alternatives, \lstinline|x+x| and \lstinline|2*x|, from the choice labelled~$A$.
A choice in a variational string is eliminated by selecting the left or right alternative. In the above example selection of the left alternative results in variant 1 and selection of the right alternative results in variant 2.
A decision is a set of selections that eliminates all choices from a variational string, as shown in table ~\ref{tab:cem1}.
%Therefore the semantics of a VP is a mapping from decisions to program variants.
Let $A$ be a choice.
We will write \myr{A} to denote the selection of the right alternative of choice $A$ and will simply write \myl{A} to denote the selection of the left alternative.

\begin{table}
    \caption{Selection of variant}\label{tab:cem1}
    \begin{center}
        \begin{tabular}{|c|c|c|}
            \hline
            Variational String & Decision & Variant \\ \hline
            \lstinline|int $A\langle$x$,$y$\rangle$ = $B\langle$1$,$2$\rangle$;| & $\lbrace\myl{A},\myl{B}\rbrace$ & \lstinline|int x = 1;| \\
            \lstinline|int $A\langle$x$,$y$\rangle$ = $B\langle$1$,$2$\rangle$;| & $\lbrace\myl{A},\myr{B}\rbrace$ & \lstinline|int x = 2;| \\
            \lstinline|int $A\langle$x$,$y$\rangle$ = $B\langle$1$,$2$\rangle$;| & $\lbrace\myr{A},\myl{B}\rbrace$ & \lstinline|int y = 1;| \\
            \lstinline|int $A\langle$x$,$y$\rangle$ = $B\langle$1$,$2$\rangle$;| & $\lbrace\myr{A},\myr{B}\rbrace$ & \lstinline|int y = 2;| \\ \hline
        \end{tabular}
    \end{center}
\end{table}

\textbf{Using the choice calculus to represent git repository - } Too obtain variational strings for a file in a repository, the changes in each commit are encoded in a new choice. Hence, each choice represents a change made to the file. All the choices created for each commit have the same dimension and therefore can be tagged to a single commit.
{\info Give an example?}
\subsection{Pattern}\label{ssec:vp}
   Patterns are represented using regular expression language. \gql supports literal character, concatenation, alternation and Kleene star{\info{yet to implement}}. The language adheres to the POSIX standard of regular expressions. 
%Following are a few examples of using regular expressions in \gql.
%\begin{description}
%\item 
%\end{description}

